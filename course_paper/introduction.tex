\section{Введение}
\label{sec:introduction}
Проблема оптимизации функций многих переменных имеет приложения во всех сферах, где используются математическое моделирование в любом виде.

Задачи линейного программирования - простейший класс таких задач, широко распространён на практике, хорошо изучен и имеет эффективные методы решения.

Однако, список задач, требующих оптимизации, выходит далеко за эти рамки.
И не каждая этих них может похвастаться настолько эффективными методами решения.

В этой работе мы рассмотрим класс задач выпуклой оптимизации, а именно оптимизация функций нескольких переменных, без ограничений.

Целью работы является сравнительный анализ градиентных методов второго порядка:

\begin{itemize}
    \item Градиентный метод Ньютона
    \item Градиентный метод Ньютона с дроблением шага (Пшеничного)
    \item Градиентный метод Бройдена - Флетчера - Гольдфарба - Шанно(БФГШ)
\end{itemize}


В частности:
\begin{itemize}
    \item найти ориентировочную вычислительную сложность на один наг алгоритма
    \item привести графическую иллюстрацию хода работы методов
    \item провести сравнение скорости сходимости на наборе показательных функций
    \item вынести рекомендации о выборе метода, в зависимости от особенностей функции
\end{itemize}

