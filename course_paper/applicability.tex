\section{Применимость методов}
\label{sec:applicability}

Градиентные методы второго порядка применимы к строго выпуклым, дважды непрерывно дифференцируемым функциям.[2]
Покажем, что это верно для выбранных функций, приведя их градиенты и матрице Гессе:

\begin{equation}
    \nabla f_1(x, y) = (4x + y,  2y + x)
    \label{eq:func_1_grad}
\end{equation}

\begin{equation}
    H_1(x, y) =
    \begin{pmatrix}
        4 & 1\\
        1 & 2
    \end{pmatrix}
    \label{eq:func_1_hesse}
\end{equation}

\begin{equation}
    \nabla f_2(x, y) = (
        4x^3,  \frac{1}{4}y^3
    )
    \label{eq:func_2_grad}
\end{equation}

\begin{equation}
    H_2(x, y) =
    \begin{pmatrix}
            12 x^2 & 0\\
            0 & \frac{3}{4}y^2
    \end{pmatrix}
    \label{eq:func_2_hesse}
\end{equation}

\begin{equation}
    \nabla f_3(x, y) = (4x^3 + \frac{x}{2},  \frac{y^3}{4} + 2y)
    \label{eq:func_3_grad}
\end{equation}

\begin{equation}
    H_3(x, y) =
    \begin{pmatrix}
        12 x^2 + \frac{1}{2} & 0\\
        0 & 3\frac{y^2}{4} + 2
    \end{pmatrix}
    \label{eq:func_3_hesse}
\end{equation}

\begin{equation}
    \nabla f_4(x, y) = (0.0015x^5 + 0.02x^3 + 2x + 5,  0.0015y^5 + 0.02y^3 + 2y + 5)
    \label{eq:func_4_grad}
\end{equation}

\begin{equation}
    H_4(x, y) =
    \begin{pmatrix}
        0.0075x^4 + 0.6x^2 + 2 & 0\\
        0 & 0.0075y^4 + 0.6y^2 + 2
    \end{pmatrix}
    \label{eq:func_4_hesse}
\end{equation}

Видим, что все функции выпуклые, и все, кроме $f_2$ - строго выпуклые.
Рассмотрим поведение методов на нестрого выпуклых функциях, а потому не будем исключать $f_2$.