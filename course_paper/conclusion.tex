\section{Заключение}
\label{sec:conclusion}
В ходе работы было показано, что квазиньютоновские методы могут показывать себя лучше Ньютоновских на определённых задачах, а также позволяют не вычислять вторые производные, что очень полезно, если функция вычисляется трудозатратно. Однако, в ряде других случаев, метод БФГШ вообще не может достичь поставленной точность, тогда как Ньютоновский метод сходится. Таким образом, имеет смысл сначала использовать метод БФГШ, а после достаточного приближения к точке, переходить на метод Ньютона.

Что касается метода Пшеничного, нам не удалось найти строго выпуклые функции, на которых он вёл бы себя отлично от метода Ньютона, но это не значит, что их нет.
